\documentclass[a4paper]{article}


\usepackage[T2A]{fontenc}
\usepackage[utf8]{inputenc}
\usepackage[russian]{babel}

\usepackage[unicode, colorlinks, linkcolor=blue, citecolor=blue]{hyperref}
\usepackage{amsmath}
\usepackage{graphicx}
\graphicspath{{pictures/}}
\DeclareGraphicsExtensions{.pdf,.png,.jpg}

\usepackage{amsmath}        % Для некоторых мат. символов
\usepackage{latexsym}       % Тоже мат символы
\usepackage{verbatim}       % Для комментирования
\usepackage{multirow}       % Для таблиц
\usepackage{multicol}       % Тоже
\usepackage{hhline}         % Для линий в таблицах



\usepackage{pgfplots}
\usepackage{tikz}

\usepackage[]{geometry}
\geometry
{
	a4paper,
	total={170mm,257mm},
	left=25mm,
	top=19mm,
	right=25mm
}

\begin{document}
	\selectlanguage{russian}
	%----------------------------------Шапка--------------------------------------
	\begin{figure}[htb]
		\begin{minipage}[c]{0.12\textwidth}
			\includegraphics[scale=0.25]{AU}
		\end{minipage}
		\hfill
		\begin{minipage}[t]{0.9\textwidth}
			{\Large\bfseries Санкт-Петербургский национальный исследовательский Академический университет имени Ж.И.~Алфёрова\\Российской академии наук}
		\end{minipage}
		\rule{164mm}{0.3mm}
	\end{figure}
	
	\begin{center}
		{\large\textbf{Рабочий протокол и отчёт по лабораторной работе № }}\\
		Свиридов Фёдор, Александр Слободнюк, Владимир Попов
	\end{center}
	\begin{center}
		\Large\bfseries{<<>>}\\
	\end{center}
	%-------------------------------------------------------------------------------
	
		\paragraph{Цель работы.} Построить амплитудно-резонансные кривые, фазовые резонансыне кривые, определить добротность маятника
		
		\paragraph{Задачи, решаемые при выполнении работы.} 
		\begin{itemize}
			\item Измерить амплитуду вынужденных колебаний, а также период данных колебаний
			\item Повторить опыты с другими коэффициентами затухания в системе, изменяя силу тока в катушке индуктивности
			\item Построить графики амплитудно-резонансных кривых
			\item С помощью графиков определить резонансную амплитуду, частоту и статическую амлитуду
			\item Определить добротность маятника при различных коэффициентах затухания
			\item Построить фазовые резонансные кривые
		\end{itemize}
		
		\paragraph{Объект исследования.} Вынужденные колебания
		
		\paragraph{Метод экспериментального исследования.}
		Измерение амплитуды вынужденных колебаний, периода и частоты
		
		\begin{comment}
	\paragraph{Рабочие формулы и исходные данные.}
		\begin{itemize}
			\item $\omega = \frac{2\pi}{T}$ - циклическая частота, $T_$ - значение периода
			\item $  \beta = \frac{-\sum t_i \ln{\frac{\alpha_{\text{ср}i}}{\alpha_0}} }{\sum t_i^2} $ - коэффициент затухания маятника Поля
			\item $  \omega_0 = \sqrt{\omega_\text{р}^2 + 2\beta^2} $ - собственная частота колебаний маятника
			\item $  Q = \frac{\alpha_\text{р}}{\alpha_\text{ст}}$ - Добротность маятника, где $\alpha_\text{р}$ - амлитуда вынужденных колебаний при резонансе, $\alpha_\text{ст}$ - статическая амплитуда
			\item $  \Psi = \arctan{\frac{-2\beta\omega}{\omega_0^2 - \omega^2}}$ - сдвиг фаз между угловым смещением маятника и вынуждающей силой
			\item $  \Delta \omega = \frac{2\pi}{T^2}\Delta T $ - погрешность циклической частоты
			\item $  \Delta \beta = \sqrt{\frac{\sum{(\Delta\epsilon_i)^2}}{(\sum t_i^2)^2} + \frac{(\sum\epsilon_i)^2}{(\sum t_i^2)^4}\sum (\Delta t_i^2)^2}$ - погрешность коэффициента затухания, где $\epsilon_i = t_i \ln{\frac{\alpha_{\text{срi}}}{\alpha_0}}$
			\item $ \Delta Q = \frac{1}{\alpha_\text{ст}}\sqrt{\Delta \alpha_\text{р}^2 + \left(\frac{\alpha_\text{р}\Delta\alpha_\text{ст}}{\alpha_\text{ст}}\right)^2}$ - погрешность добротности маятника
		\end{itemize}
		\end{comment}
		
		
		
		\paragraph{Результаты прямых измерений и их обработки (таблицы, примеры расчетов)}$ $\\
		\begin{table}[h!]
			\centering
			\caption{...}
			\begin{tabular}{| m{1.5em} | m{2em} | m{1.5em} |  m{3em} | m{1.5em} | m{1.5em} | m{2em} | m{1.5em} | m{3em}| m{1.5em} | m{1.5em} | m{2em} | m{1.5em}| m{3em} |}
				\hhline{----~----~----}
				U,В & T,с & $\alpha$ & $\omega$,с$^{-1}$ & & U,В & T,с & $\alpha$ & $\omega$,с$^{-1}$ & & U,В & T,с & $\alpha$ & $\omega$,с$^{-1}$\\
				\hhline{----~----~----}
				1.32 &	3.00 &	4.5 & $2.094 \pm 0.007$ & & 1.32 & 2.93 &	5.0 &	$2.144 \pm 0.007$ & & 1.32 & 2.97 & 5.0 & $2.116 \pm 0.007$ \\
				\hhline{----~----~----}
				1.72 &	2.09 &	6.0	& $3.006 \pm 0.014$ & &	1.51 &	2.48 &	6.0	& $2.534 \pm 0.011$ & &	1.80 &	1.90 &	7.0 &	$3.307 \pm 0.017$
				\\
				\hhline{----~----~----}
				2.09 &	1.60 &	8.0 &	$3.927 \pm 0.025$ & &	1.92 &	1.78 &	7.0 &	$3.53 \pm 0.19$ &	 & 2.23 &	1.47 &	9.5 &	$ 4.27 \pm  0.04$
				\\
				\hhline{----~----~----}
				2.47 &	1.29 &	9.5 &	$4.87 \pm 0.04$ & &	2.36 &	1.35 &	10.0 &	$4.65 \pm 0.03$ & &	2.81 &	1.11 &	15.0 &	$5.66 \pm 0.05 $
				\\
				\hhline{----~----~----}
				2.68 &	1.19 &	12.0 &	$5.28 \pm 0.04$ & &	2.60 &	1.21 &	12.0 &	$5.19 \pm 0.04$ & &	3.03 &	1.01	& >24 &	$6.221 \pm 0.06$
				\\
				\hhline{----~----~----}
				
				2.83 &	1.10 &	15.5 &	$5.71 \pm 0.05$ & &	2.83 & 	1.08	& 17.0 &	$5.82 \pm 0.05$ &	 & 4.42 &	0.64 &	4.0 &	$9,82 \pm 0.15$
				\\
				\hhline{----~----~----}
				
				3.00 &	1.01 &	>24 &	$6.22 \pm 0.06$& &	3.01 &	1.00 &	>24 &	$6.28 \pm 0.06$ &	& 4.17 &	0.69 &	5.0 &	$9.11 \pm 0.13$
				\\
				\hhline{----~----~----}
				
				4.41 &	0.64 &	3.5 &$	9.82 \pm 0.15$ & &	4.40 &	0.65 &	3.5 &	$9.67 \pm 0.15$ & &	3.87 &	0.74 &	8.0 &	$8.49 \pm 0.11$
				\\
				\hhline{----~----~----}
				
				4.14 &	0.65 &	5.0 &	$9.67 \pm 0.15$ & &	4.08 &	0.71 &	4.5 &	$8.85 \pm 0.13$ & &	3.47 &	0.84 &	14.0 &	$7.48 \pm 0.09$
				\\
				\hhline{----~----~----}
				
				3.86 &	0.69 &	7.0 &	$9.11 \pm 0.13$ & &	3.81 &	0.75 &	7.5 &	$8.38 \pm 0.11$ & &	3.09 &	0.98 &	>24 &	$6.41 \pm 0.06$
				\\
				\hhline{----~----~----}
				
				3.61 &	0.77 &	11.0 &	$8.16 \pm 0.11$ & &	3.49 &	0.85 &	13.5 &	$7.39 \pm 0.09 $
				\\
				\hhline{----~----~~~~~~}
				
				3.36 &	0.86 &	17.5 &	$7.31 \pm 0.08$ & &	3.25 &	0.91 &	21.0 &	$6.90 \pm 0.08$
				\\
				
				\hhline{----~----~~~~~~}
				
				3.25 &	0.94 &	21.0 &	$6.68 \pm 0.07$ & &	3.02 &	1.00 &	>24 &	$6.28 \pm 0.06$ 
				\\
				
				\hhline{----~----~~~~~~}
				
				3.03 &	1.00 &	>24 &	$6.28 \pm 0.06$
				\\
				
				\hhline{----~~~~~~~~~~~}
				
				
			\end{tabular}

		\end{table}
		
		
		
		По данным таблиц для 0A, 4A и 5A соответственно:
		$$
		\alpha_\text{р1} = 33 \pm 2, \;\; w_\text{р1} = 6.22, \;\; \alpha_\text{ст} = 6 \pm 1
		$$
		$$
		\alpha_\text{р2} = 26 \pm 2, \;\; w_\text{р2} = 6.40, \;\; \alpha_\text{ст} = 6 \pm 1
		$$
		$$
		\alpha_\text{р3} = 29 \pm 2, \;\; w_\text{р3} = 6.39, \; \;
		\alpha_\text{ст} = 6 \pm 1
		$$
		
		\paragraph{Расчет результатов косвенных измерений.}$ $\\
		
		Коэффициенты затухания известны из лабораторной работы №1:
		
		$$
		\beta_1 = 0.147 \pm 0.007
		$$
		
		$$
		\beta_2 = 0.135 \pm 0.006
		$$
		
		$$
		\beta_3 = 0.135 \pm 0.008
		$$
		
		Вычислим собственную частоту колебаний маятника по формуле $w_0 = \sqrt{w_\text{р}^2 + 2\beta^2}$
		$$
		\omega_\text{03} \approx 6.22
		$$
		$$
		\omega_\text{02} \approx 6.4
		$$
		$$
		\omega_\text{01} \approx 6.39
		$$
		
		Вычислим добротность маятника по формуле:
		$Q = \frac{\alpha_\text{р}}{\alpha_\text{ст}}$\\
		
		и его абсолютную погрешность по формуле: $\displaystyle \Delta Q = \frac{1}{\alpha_\text{ст}}\sqrt{\Delta \alpha_\text{р}^2 + \left(\frac{\alpha_\text{р}\Delta\alpha_\text{ст}}{\alpha_\text{ст}}\right)^2}$
		$$
		Q_3 \approx 5.5 \;\; \Delta Q_3 \approx 1.0
		$$
		$$
		Q_2 \approx 4.4 \;\; \Delta Q_2 \approx 0.8
		$$
		$$
		Q_1 \approx 4.8 \;\; \Delta Q_1 \approx 0.9
		$$
		
	\end{document}

