\documentclass[a4paper]{article}

\usepackage[T2A]{fontenc}
\usepackage[utf8]{inputenc}
\usepackage[russian]{babel}

\usepackage[unicode, colorlinks, linkcolor=blue, citecolor=blue]{hyperref}
\usepackage{amsmath}
\usepackage{graphicx}
\graphicspath{{pictures/}}
\DeclareGraphicsExtensions{.pdf,.png,.jpg}

\usepackage{pgfplots}
\usepackage{tikz}
\usepackage{verbatim}       % Для комментирования
\usepackage{multirow}       % Для таблиц
\usepackage{multicol}       % Тоже
\usepackage{hhline}         % Для линий в таблицах


\usepackage[]{geometry}
\geometry
{
	a4paper,
	total={170mm,257mm},
	left=25mm,
	top=19mm,
	right=25mm
}

\begin{document}
	\selectlanguage{russian}
	%----------------------------------Шапка--------------------------------------
	\begin{figure}[htb]
		\begin{minipage}[c]{0.12\textwidth}
			\includegraphics[scale=0.25]{AU}
		\end{minipage}
		\hfill
		\begin{minipage}[t]{0.9\textwidth}
			{\Large\bfseries Санкт-Петербургский национальный исследовательский Академический университет имени Ж.И.~Алфёрова\\Российской академии наук}
		\end{minipage}
		\rule{164mm}{0.3mm}
	\end{figure}
	
	\begin{center}
		{\large\textbf{Рабочий протокол и отчёт по лабораторной работе № 2}}\\
		Свиридов Фёдор, Александр Слободнюк, Владимир Попов
	\end{center}
	\begin{center}
		\Large\bfseries{<<Маятник Поля>>}\\
	\end{center}
	%-------------------------------------------------------------------------------
	{\parindent=0pt\textbf{Цель работы.}}\\
	Изучить понятие о резонансной частоте и вынужденных колебаниях.\\
	
	{\parindent=0pt\textbf{Задачи, решаемые при выполнении работы.}}
	\begin{enumerate}
		\item Начать колебания и записать процесс на камеру
		\item Измерить период и амплитуду, соответствующую данной частоте колебаний
		\item Рассчитать значение частоты и амплитуды при резонансе
		\item Рассчитать фазовый сдвиг
	\end{enumerate}

	{\parindent=0pt\textbf{Объект исследования.}}\\
	Вынужденные колебания, резонанс\\
	
	{\parindent=0pt\textbf{Метод экспериментального исследования.}}
	\begin{enumerate}
		\item Измерение амплитуды установившихся колебаний
		\item Измерение периода колебаний\\
	\end{enumerate}

	{\parindent=0pt\textbf{Рабочие формулы.}}\\
	
	\begin{equation}
		\omega = \frac{2\pi}{T}
	\end{equation}
		\begin{equation}
		 \omega_0 = \sqrt{\omega_\text{р}^2 + 2\beta^2}
	\end{equation}
			\begin{equation}
		 \Psi = \arctan{\frac{-2\beta\omega}{\omega_0^2 - \omega^2}}
	\end{equation}

\begin{equation}
	Q = \frac{\alpha_\text{р}}{\alpha_\text{ст}}
\end{equation}
, где $ \omega_0$ - собственная частота колебательной системы\\

{\parindent=0pt\textbf{Результаты прямых измерений и их обработки.}}\\

\begin{figure}[htb]
	\begin{minipage}[b]{0.3\textwidth}
		\centering{$I=0\;A$}
			\begin{tabular}{|c|c|c|c|}
				\hline
				$U$, В&$\alpha$, град.& $T$, c& $\omega,\;c^{-1}$  \\
				\hline
				1,3&  22,5&   3,3&1,9\\
				
				2,1& 32,5 &1,5   &4,2\\
				
				2,8& 75 &  1 &6,2\\
				
				3& 120 &  0,9 &6,9\\
				
				4,3& 17,5 &  0,6 &10,5\\
				\hline
			\end{tabular}
	\end{minipage}
\hfill
	\begin{minipage}[b]{0.3\textwidth}
		\centering{$I=4\;A$}
		\begin{tabular}{|c|c|c|c|}
			\hline
			$U$, В&$\alpha$, град.& $T$, c& $\omega,\;c^{-1}$  \\
			\hline
			1,3&  25&   2,9&2,2\\
			
			2,1& 45&1,5   &4,2\\
			
			2,8& 85 &  1,1 &5,7\\
			
			3& 120 &  1 &6,2\\
			
			4,3& 17,5 &  0,6 &10,5\\
			\hline
		\end{tabular}
\end{minipage}
\hfill
	\begin{minipage}[b]{0.3\textwidth}
		\centering{$I=5\;A$}
		\begin{tabular}{|c|c|c|c|}
			\hline
			$U$, В&$\alpha$, град.& $T$, c& $\omega,\;c^{-1}$  \\
			\hline
			1,3&  25&   3&2,1\\
			
			2,1& 47,5 &1,4   &4,5\\
			
			2,8& 75 &  1,1 &5,7\\
			
			3& 120 &  1 &6,2\\
			
			4,3& 20 &  0,7 &9\\
			\hline
		\end{tabular}
\end{minipage}
\end{figure}


		\begin{center}
		\begin{tikzpicture}[scale=0.9]
			\begin{axis}[ylabel={$\alpha,\;\mbox{град.}$}, xlabel={$\omega,\;c^{-1}$}, xmajorgrids, ymajorgrids, minor tick num = 1]
					\legend{$I=0\;A$}
				\addplot[black, point meta=explicit symbolic,mark=*, mark size=1.5pt, only marks] 
				coordinates {
					(1.9,22.5 )
					
					(4.2,32.5)
					
					(6.2, 75 )
					
					( 6.9,120)
					
					(10.5,17.5)
				};
			\end{axis}
		\end{tikzpicture}
	\end{center}


	\begin{center}
		\begin{tikzpicture}[scale=0.9]
			\begin{axis}[ylabel={$\alpha,\;\mbox{град.}$}, xlabel={$\omega,\;c^{-1}$}, xmajorgrids, ymajorgrids, minor tick num = 1]
				\legend{$I=4\;A$}
				\addplot[black, point meta=explicit symbolic,mark=*, mark size=1.5pt, only marks] 
				coordinates {
			(2.2,25)

(4.2,45)

 (  5.7,85)

 ( 6.2,120)

(  10.5,17.5 )
				};
			\end{axis}
		\end{tikzpicture}
	\end{center}



	\begin{center}
		\begin{tikzpicture}[scale=0.9]
			\begin{axis}[ylabel={$\alpha,\;\mbox{град.}$}, xlabel={$\omega,\;c^{-1}$}, xmajorgrids, ymajorgrids, minor tick num = 1]
				\legend{$I=5\;A$}
				\addplot[black, point meta=explicit symbolic,mark=*, mark size=1.5pt, only marks] 
				coordinates {
			  (  2.1,25)
 (4.5,47.5)
  (  5.7,75)
  (  6.2,120)
 ( 9, 20)
				};
			\end{axis}
		\end{tikzpicture}
	\end{center}


{\parindent=0pt\textbf{Расчёт результатов косвенных измерений.}}\\
К сожалению, полученных данных не хватает для нормальной экстраполяции полученных точек, поэтому для оценки других физических величин будем считать, что для всех трёх опытов:
$$\omega_{p}\approx6,3\;(c^{-1})$$
$$\alpha_{p}=120\;\mbox{град.}$$
Из лабораторной работы №1 знаем, что $\beta_1 \approx 0,1$; $\beta_2 \approx 0,09$; $\beta_3 \approx 0,14$. Тогда по формуле (2) находим собственную частоту колебательной системы:
$$ \omega_{01}=\sqrt{(6,3)^2+2\cdot(0,1)^2}\approx6,3\;(c^{-1})$$
$$ \omega_{02}=\sqrt{(6,3)^2+2\cdot(0,09)^2}\approx6,3\;(c^{-1})$$
$$ \omega_{03}=\sqrt{(6,3)^2+2\cdot(0,14)^2}\approx6,3\;(c^{-1})$$
Так как данные были получены с очень маленькой точностью, и поэтому почти не отличаются, для оценки сдвига фазы $\Psi$ положим $\beta=0,1$; $\omega_0=6,3\;(c^{-1})$ и возьмём средние значения  частот из трёх опытов.

	\begin{center}
	\begin{tabular}{|c|c|c|c|}
	\hline
	$\beta$&$\omega_0,\;c^{-1}$&$\omega,\;c^{-1}$&$\Psi$, рад.  \\
			\hline
&  				&   2,1&-0,01\\

&  				&4,3   &-0,04\\

0,1& 6,3  &  5,9 &-0,24\\

&  				&  6,4&-2,35\\

&  				&  10 &-3,11\\
\hline
	\end{tabular}
	\end{center}

\begin{center}
\begin{tikzpicture}[scale=1]
	\begin{axis}[nodes near coords,enlargelimits=0.2, ylabel={$\Psi,\;\mbox{рад.}$}, xlabel={$\omega,\;c^{-1}$}, xmajorgrids, ymajorgrids, minor tick num = 1]
		\legend{,,$\arctan{\frac{-2\beta\omega}{\omega_0^2-\omega^2}}$}
		\addplot[black, point meta=explicit symbolic,mark=*, mark size=1.5pt, only marks] 
coordinates {
	(2.1,-0.01)
	
	(4.3   ,-0.04)
	
	(5.9 ,-0.24)
	
	(6.4,-2.35)
	
	(10 ,-3.11)
};

		\addplot[blue, point meta=explicit symbolic,mark=*, mark size=1pt, only marks] 
coordinates { (6.3,-1.571)[$\qquad-\frac{\pi}{2}$]};

	\addplot [red,point meta=explicit symbolic, line width=1pt, domain=0:6.299999]{rad(atan((-2*0.1*x)/(39.69-x^2)))};
	\addplot [red,point meta=explicit symbolic, line width=1pt, domain=6.3001:11]{rad(atan((-2*0.1*x)/(39.69-x^2)))-3.14};
	
	
	\end{axis}
\end{tikzpicture}
\end{center}



	\end{document}
