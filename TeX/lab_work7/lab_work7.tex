\documentclass[a4paper]{article}

\usepackage[T2A]{fontenc}
\usepackage[utf8]{inputenc}
\usepackage[russian]{babel}

\usepackage[unicode, colorlinks, linkcolor=blue, citecolor=blue]{hyperref}
\usepackage{amsmath}
\usepackage{graphicx}
\graphicspath{{pictures/}}
\DeclareGraphicsExtensions{.pdf,.png,.jpg}
\usepackage{multirow}

\usepackage{pgfplots}
\usepackage{tikz}

\usepackage[]{geometry}
\geometry
{
	a4paper,
	total={170mm,257mm},
	left=25mm,
	top=19mm,
	right=25mm
}

\begin{document}
	\selectlanguage{russian}
	%----------------------------------Шапка--------------------------------------
	\begin{figure}[htb]
		\begin{minipage}[c]{0.12\textwidth}
			\includegraphics[scale=0.25]{AU}
		\end{minipage}
		\hfill
		\begin{minipage}[t]{0.9\textwidth}
			{\Large\bfseries Санкт-Петербургский национальный исследовательский Академический университет имени Ж.И.~Алфёрова\\Российской академии наук}
		\end{minipage}
		\rule{164mm}{0.3mm}
	\end{figure}
	
	\begin{center}
		{\large\textbf{Рабочий протокол и отчёт по лабораторной работе № 7}}\\
		Свиридов Фёдор, Александр Слободнюк, Владимир Попов
	\end{center}
	\begin{center}
		\Large\bfseries{<<Проверка закона Бойля-Мариотта>>}\\
	\end{center}
	%-------------------------------------------------------------------------------
	
	{\parindent=0pt\textbf{Цель работы.}}\\
	Исследовать изотермический процесс\\
	
	{\parindent=0pt\textbf{Исходные данные.}}\\
	Пусть $S$ - площадь цилиндра, а $l$ - высота, на которой находится поршень. Применяя модель\\ идеального газа для воздуха, получаем:
	$$ PV=\frac{m}{\mu} RT$$
	$$ P=\frac{\rho V_0}{\mu}RT\cdot\frac{1}{V}$$
	$$ P=\frac{\rho S l_0}{\mu}RT\cdot\frac{1}{S l}$$
	$$ P=\frac{\rho  l_0}{\mu}RT\cdot\frac{1}{ l}$$
	При $l_0=35$ (см) у нас $P=P_0$ ($P_0$ - атмосферное давление), поэтому в итоге:
	$$\Delta P(l)=A\cdot \frac{1}{l} - P_0$$
	, где $\Delta P$ - давление над атмосферным; $A = \frac{\rho  l_0}{\mu}RT$.\\
	\parindent=0pt
	
	Таким образом, ожидаемый коэффициент $A$ для $T=28\;^\circ C$ равен
	$$ A=\frac{1,2\cdot35\cdot8,31\cdot302}{29\cdot10^{-3}\cdot10^3}\approx3635\;\left( \mbox{кПа}\cdot\mbox{см}\right) $$\\
	
	{\parindent=0pt\textbf{Результаты прямых измерений.}}\\
	
	\begin{figure}[htb]		
		\begin{minipage}[b]{0.3\textwidth}
			
					\begin{tabular}{c|c}
					\multicolumn{2}{c}{$T=28\;^{\circ} C$}\\
					\hline
					$l$, см&$\Delta P$, кПа  \\
					\hline
					35&  0\\
					
					34& 2,2 \\
					
					33&  4,7\\
					
					32&  7,9\\
					
					31&11,5  \\
					
					30& 15,4 \\
					
					29&  20,5\\
					
					28&  26,0\\
					
					27&32,0\\
					\multicolumn{2}{c}{}
				\end{tabular}
		\end{minipage}
	\hfil
	\begin{minipage}[b]{0.3\textwidth}
		\begin{tabular}{c|c}
			\multicolumn{2}{c}{$T=33\;^{\circ} C$}\\
			\hline
			$l$, см&$\Delta P$, кПа  \\
			\hline
			35&  0\\
			
			34& 2,4 \\
			
			33&  5,3\\
			
			32&  8,7\\
			
			31&12,4  \\
			
			30& 16,6 \\
			
			29& 21,3\\
			
			28&  27,5\\
			
			27&33,2\\
			
			26&41,0
		\end{tabular}
	\end{minipage}
	\hfil
	\begin{minipage}[b]{0.3\textwidth}

		\begin{tabular}{c|c}
			\multicolumn{2}{c}{$T=39\;^{\circ} C$}\\
			\hline
			$l$, см&$\Delta P$, кПа  \\
			\hline
			35&  0\\
			
			34& 1,3 \\
			
			33&  4,0\\
			
			32&  7,0\\
			
			31&11,5 \\
			
			30& 15,4 \\
			
			29&  19,8\\
			
			28&  25,8\\
			
			27&34,0\\
			
			26&43,0
		\end{tabular}
	\end{minipage}
\end{figure}
	
	\newpage
	{\parindent=0pt\textbf{Обработка результатов.}}\\
	\begin{figure}[htb]
		\centering{\textbf{Изотермы при разных температурах}}
	\begin{center}
	\begin{minipage}[c]{0.32\textwidth}
		 \begin{tikzpicture}[scale=0.6]
			\begin{axis}[xlabel={$l$, см}, ylabel={$\Delta P$, кПа}, ymajorgrids, minor tick num = 1]
				\legend{,$ 3797.97\cdot\frac{1}{x}-110.02$};
				\addplot[black, point meta=explicit symbolic,mark=*, mark size=1.3pt, only marks] 
				coordinates {
					(	35	,	0	)
					(	34	,	2.2	)
					(	33	,	4.7	)
					(	32	,	7.9	)
					(	31	,	11.5	)
					(	30	,	15.4	)
					(	29	,	20.5	)
					(	28	,	26	)
					(	27	,	32	)
				};
				\addplot [draw=red][domain=24:36]{3797.97/x-110.02};
				
			\end{axis}
			\end{tikzpicture}
		
		{\parindent=2.25cm $T=28\;^{\circ} C$}
	\end{minipage}
\hfill
	\begin{minipage}[c]{0.32\textwidth}
		 \begin{tikzpicture}[scale=0.6]
			\begin{axis}[xlabel={$l$, см}, ylabel={$\Delta P$, кПа}, ymajorgrids, minor tick num = 1]
				\legend{,$ 4121.88\cdot\frac{1}{x}-119.52$};
					\addplot[black, point meta=explicit symbolic,mark=*, mark size=1.3pt, only marks] 
		coordinates {
			(	35	,	0	)
			(	34	,	2.4	)
			(	33	,	5.3	)
			(	32	,	8.7	)
			(	31	,	12.4	)
			(	30	,	16.6	)
			(	29	,	21.3	)
			(	28	,	27.5	)
			(	27	,	33.2	)
			(	26	,	41	)
		};
		\addplot [draw=red][domain=24:36]{4121.88/x-119.521};
	\end{axis}
\end{tikzpicture}

{\parindent=2.25cm $T=33\;^{\circ} C$}
	\end{minipage}
\hfill
	\begin{minipage}[c]{0.32\textwidth}
	\begin{tikzpicture}[scale=0.6]
		\begin{axis}[xlabel={$l$, см}, ylabel={$\Delta P$, кПа}, ymajorgrids, minor tick num = 1]
			\legend{,$ 4309.08\cdot\frac{1}{x}-126.37$};
			\addplot[black, point meta=explicit symbolic,mark=*, mark size=1.3pt, only marks] 
coordinates {
	(	35	,	0	)
	(	34	,	1.3	)
	(	33	,	4	)
	(	32	,	7	)
	(	31	,	11.5	)
	(	30	,	15.4	)
	(	29	,	19.8	)
	(	28	,	25.8	)
	(	27	,	34	)
	(	26	,	43	)
};
\addplot [draw=red][domain=24:36]{4309.08/x-126.374};
		\end{axis}
	\end{tikzpicture}

{\parindent=2.25cm $T=39\;^{\circ} C$}
\end{minipage}
\end{center}
\end{figure}




\begin{figure}[htb]
	\begin{center}
		\begin{tikzpicture}[scale=1.1]
			\begin{axis}[xlabel={$l$, см}, ylabel={$\Delta P$, кПа}, ymajorgrids, minor tick num = 1]
				\legend{,$T=28\;^{\circ} C$, ,$T=33\;^{\circ} C$, ,$T=39\;^{\circ} C$}
				
				\addplot[red, point meta=explicit symbolic,mark=square*, mark size=0.9pt, only marks] 
				coordinates {
					(	35	,	0	)
					(	34	,	2.2	)
					(	33	,	4.7	)
					(	32	,	7.9	)
					(	31	,	11.5	)
					(	30	,	15.4	)
					(	29	,	20.5	)
					(	28	,	26	)
					(	27	,	32	)
				};
				\addplot [draw=red][domain=24:36]{3797.97/x-110.02};
				
				%-------------------------------------------------------------------------------
				\addplot[blue, point meta=explicit symbolic,mark=triangle*, mark size=1.2pt, only marks] 
				coordinates {
					(	35	,	0	)
					(	34	,	2.4	)
					(	33	,	5.3	)
					(	32	,	8.7	)
					(	31	,	12.4	)
					(	30	,	16.6	)
					(	29	,	21.3	)
					(	28	,	27.5	)
					(	27	,	33.2	)
					(	26	,	41	)
				};
				\addplot [draw=blue][domain=24:36]{4121.88/x-119.521};
				%-------------------------------------------------------------------------------
				\addplot[black, point meta=explicit symbolic,mark=*, mark size=0.9pt, only marks] 
				coordinates {
					(	35	,	0	)
					(	34	,	1.3	)
					(	33	,	4	)
					(	32	,	7	)
					(	31	,	11.5	)
					(	30	,	15.4	)
					(	29	,	19.8	)
					(	28	,	25.8	)
					(	27	,	34	)
					(	26	,	43	)
				};
				\addplot [draw=black][domain=24:36]{4309.08/x-126.374};
			\end{axis}
		\end{tikzpicture}
	\end{center}
\end{figure}

{\parindent=0pt\textbf{Выводы и анализ результатов.}}\\
Мы измерили 

	\end{document}
