\documentclass[a4paper]{article}

\usepackage[T2A]{fontenc}
\usepackage[utf8]{inputenc}
\usepackage[russian]{babel}

\usepackage{csvsimple}
\usepackage[unicode, colorlinks, linkcolor=blue, citecolor=blue]{hyperref}
\usepackage{amsmath}
\usepackage{graphicx}
\graphicspath{{pictures/}}
\DeclareGraphicsExtensions{.pdf,.png,.jpg}

\usepackage[]{geometry}
\geometry
{
	a4paper,
	total={170mm,257mm},
	left=25mm,
	top=19mm,
	right=25mm
}

\begin{document}
	\selectlanguage{russian}
	%----------------------------------Шапка--------------------------------------
	\begin{figure}[htb]
		\begin{minipage}[c]{0.12\textwidth}
			\includegraphics[scale=0.25]{AU}
		\end{minipage}
		\hfill
		\begin{minipage}[t]{0.9\textwidth}
			{\Large\bfseries Санкт-Петербургский национальный исследовательский Академический университет имени Ж.И.~Алфёрова\\Российской академии наук}
		\end{minipage}
		\rule{164mm}{0.3mm}
	\end{figure}
	
	\begin{center}
		{\large\textbf{Рабочий протокол и отчёт по лабораторной работе №4 }}\\
		Свиридов Фёдор, Александр Слободнюк, Владимир Попов
	\end{center}
	\begin{center}
		\Large\bfseries{<<Свободные затухающие колебания в параллельном LC-контуре>>}\\
	\end{center}
	%-------------------------------------------------------------------------------
	

\csvreader[tabular=c|c, table head=\hline $X$ & $Y$\\\hline]
{1.csv}{X=\X,Y=\Y}
{  \X & \Y}

	\end{document}
