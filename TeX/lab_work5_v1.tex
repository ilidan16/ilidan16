\documentclass{article}

\usepackage[T2A]{fontenc}
\usepackage[utf8]{inputenc}
\usepackage[russian]{babel}

\usepackage{amsmath}

\usepackage[unicode, colorlinks, linkcolor=blue]{hyperref}

\usepackage{graphicx}
\graphicspath{{pictures/}}
\DeclareGraphicsExtensions{.pdf,.png,.jpg}

\usepackage{pgfplots}


\begin{document}
%----------------------------------Шапка---------------------------------------------
\begin{center}
	\includegraphics[scale=0.25]{AU}\\
	{\Large\bfseries Санкт-Петербургский национальный исследовательский Академический университет имени Ж.И.~Алфёрова Российской академии наук}
\end{center}

\begin{center}
	Свиридов Фёдор, Александр Слободнюк, Владимир Попов
\end{center}
\rule{12cm}{0.4mm}
\begin{center}
	{\large\textbf{Рабочий протокол и отчёт по лабораторной работе № 5}}
\end{center}
%--------------------------------------------------------------------------------------
\paragraph{Цель работы.}
Вычислить момент инерции маятника Обербека

\paragraph{Задачи, решаемые при выполнении работы.}
\begin{itemize}
	\item Измерить массы грузов
	\item Измерить диаметр шкива
	\item Измерить высоту, с которой опускаются грузы
	\item Измерить время, за которое опускаются грузы с различной массой (маятник без грузов)
	\item Найти зависимость $\varepsilon(m)$ и с помощью экстраполяции определить $m_o$
	\item Вычислить момент инерции маятника без грузов
	\item Вычислить момент инерции маятника с грузами при различных расстояниях $r$
	\item Сделать выводы
\end{itemize}

\paragraph{Объект исследования.}
Момент инерции $I$

\paragraph{Метод экспериментального исследования.}
Измерение момента инерции

 \paragraph{Рабочие формулы и исходные данные.}\hypertarget{formuls}{}
 \begin{equation}
 	\fbox{$I=\frac{g D^2}{8x}(m-m_0)t^2$}
 \end{equation}

\begin{equation}
	\varepsilon=\frac{4x}{Dt^2}
\end{equation}

\begin{equation}
	\Delta \varepsilon=\sqrt{\frac{16}{D^2t^4}\Delta x^2+\frac{16x^2}{D^4t^4}\Delta D^2+\frac{64x^2}{D^2t^6}\Delta t^2}
\end{equation}
где $D$ - диаметр шкива; $x$ - высота, с которой спускается груз; $ t $ -время спуска груза; $m$ - масса груза; $m_0$ - масса груза, которая компенсирует момент силы трения (определяется косвенно)



\paragraph{Результаты прямых измерений и их обработки.}
\begin{itemize}
	\item $ D=63,37\;\mbox{мм} $
	\item $ x=45,5\;\mbox{см} $
	\item Средняя масса грузов, которые крепятся на стержни:\\
	$ \overline{m}_1 = 115,16\;\mbox{г}$
	\item Маятник без грузов на стержнях
	
	\begin{tabular}{c|c}
		
		$m$, г& $t$, с \\
		\hline
		46,45& 5,356 \\
		
		95,75& 3,847 \\
		
		145,05& 3,140 \\
		
		194,35& 2,785 \\
	\end{tabular}
\item Маятник с грузами $\overline{m}_1$ на стержнях, расположенные на расстояние $r$ от оси вращения:

\begin{tabular}{c|c|c}

	$r$, см&$m$, г& $t$, с \\
	\hline
	27,75&95,75  &6,941  \\
	22,75&95,75  &6,142  \\
	17,75&95,75  &5,063  \\
	
\end{tabular}

где $m$ - масса спускаемого груза
\end{itemize}

\paragraph{Погрешности измерений.}
\begin{itemize}
	\item $\Delta m = 0,01\;\mbox{г}$
	\item $\Delta t = 0,001\;\mbox{c}$
	\item $\Delta x = 1\;\mbox{см}$
	\item $\Delta D = 0,02\;\mbox{мм}$
	\item $\Delta r = 0,5\;\mbox{см}$
\end{itemize}

\paragraph{Расчет результатов косвенных измерений.}

\begin{itemize}
	\item {\bf Вычисление $\bf m_0$}
	
	
\begin{itemize}
	\item Пользуясь \hyperlink{formuls}{формулой (2)} находим $\varepsilon$:
	
	$$\varepsilon_1=\frac{4\cdot0,455}{63,37\cdot10^{-3}\cdot(5,356)^2}\approx1\;\left( \mbox{с}^{-2} \right) $$
	$$\varepsilon_2=\frac{4\cdot0,455}{63,37\cdot10^{-3}\cdot(3,847)^2}\approx1,94\;\left( \mbox{с}^{-2} \right) $$
	$$\varepsilon_3=\frac{4\cdot0,455}{63,37\cdot10^{-3}\cdot(3,140)^2}\approx2,91\;\left( \mbox{с}^{-2} \right) $$
	$$\varepsilon_4=\frac{4\cdot0,455}{63,37\cdot10^{-3}\cdot(2,785)^2}\approx3,7\;\left( \mbox{с}^{-2} \right) $$

	\item Пользуясь \hyperlink{formuls}{формулой (3)} находим погрешность $\Delta \varepsilon$:
	
	\begin{tabular}{c|c}
		№ & $\Delta \varepsilon,\;\mbox{c}^{-2}$ \\
		\hline
		1 &  0,02\\
		
		2 &  0,04\\
		
		3 &  0,06\\
		
		4 &  0,08\\
	\end{tabular}\\

\item В итоге:

	\begin{tabular}{ c | c | c}
		№	& $m$, г& $\varepsilon$, $\mbox{с}^{-2}$ \\
		\hline
		1 & $46,45\pm0,01 $ &  $  1\pm0,02  $\\
		
		2& $ 95,75\pm0,01$ & $ 1,94\pm0,04  $ \\
		
		3& $145,05 \pm0,01 $&  $ 2,91 \pm0,06 $\\
		
		4& $ 194,35\pm0,01 $& $  3,7 \pm0,08$
	\end{tabular}\\

\item С помощью метода наименьших квадратов экстраполируем $m_0$:
$$ f(k, b)=\left( \varepsilon_1-(km_1+b)\right) ^2+\left( \varepsilon_2-(km_2+b)\right) ^2+\left( \varepsilon_3-(km_3+b)\right) ^2+\left( \varepsilon_4-(km_4+b)\right) ^2$$
\begin{equation*}
	\begin{cases}
		\frac{\partial f}{\partial k}=-2m_1( \varepsilon_1-km_1-b) - 2m_2( \varepsilon_2-km_2-b) - 2m_3( \varepsilon_3-km_3-b) - 2m_4( \varepsilon_4-km_4-b) =0\\
		\frac{\partial f}{\partial b}=-2( \varepsilon_1-km_1-b) - 2( \varepsilon_2-km_2-b) - 2( \varepsilon_3-km_3-b) - 2( \varepsilon_4-km_4-b) =0
		
	\end{cases}
\end{equation*}

\begin{equation*}
	\begin{cases}
		2(m_1\varepsilon_1+m_2\varepsilon_2+m_3\varepsilon_3+m_4\varepsilon_4)-2k(m_1^2+m_2^2+m_3^2+m_4^2)-2b(m_1+m_2+m_3+m_4)=0\\
		2(\varepsilon_1 + \varepsilon_2 + \varepsilon_3 + \varepsilon_4)-2k(m_1+m_2+m_3+m_4)-8b=0
	\end{cases}
\end{equation*}

\begin{equation*}
	\begin{cases}
		(2,75\pm0.03)-(0,140274\pm0,000019)k-(0,96320\pm0,00008)b=0\\
		(19,1\pm0,4)-(0,96320\pm0,00008)k-8b=0
	\end{cases}
\end{equation*}

\begin{equation*}
	\begin{cases}
	k = 18,5297\pm2,3345\\
	b=0,15652\pm0,32460
\end{cases}
\end{equation*}

$$ m_0=-\frac{b}{k}$$



 \begin{equation*}
	\fbox{$ m_0= ({-}0,008\pm0,018)\;\mbox{кг}$}
\end{equation*}
\end{itemize}
\end{itemize}


\end{document}





