\documentclass{article}

\usepackage[T2A]{fontenc}
\usepackage[utf8]{inputenc}
\usepackage[russian]{babel}

\usepackage{amsmath}

\usepackage[unicode, colorlinks, linkcolor=blue]{hyperref}

\usepackage{graphicx}
\graphicspath{{pictures/}}
\DeclareGraphicsExtensions{.pdf,.png,.jpg}

\usepackage{pgfplots}
\usepackage{tikz}

\begin{document}
%----------------------------------Шапка------------------------------------
\begin{center}
	\includegraphics[scale=0.25]{AU}\\
	{\Large\bfseries Санкт-Петербургский национальный исследовательский Академический университет имени Ж.И.~Алфёрова Российской академии наук}
\end{center}

\begin{center}
	Свиридов Фёдор, Александр Слободнюк, Владимир Попов
\end{center}
\rule{12cm}{0.4mm}
\begin{center}
	{\large\textbf{Рабочий протокол и отчёт по лабораторной работе № 3}}
\end{center}
%-----------------------------------------------------------------------------

\paragraph{Цель работы.}
Определение средней длины свободного пробега молекул воздуха; определение эффективного диаметра молекул воздуха

\paragraph{Задачи, решаемые при выполнении работы.}
\begin{enumerate}
	\item Узнать температуру в помещении и атмосферное давление в данный момент времени
	\item Измерить нужное для посчетов время, за которое в мензурку поступало 100мл воды
	\item Измерить разницу высот жидкости в колбе до и после спучка воды
\end{enumerate}

\paragraph{Объект исследования.}
Молекулы воздуха

\paragraph{Метод экспериментального исследования.}
Измерение времени и уровней жидкости в колбе

 \paragraph{Рабочие формулы и исходные данные.}\hypertarget{formuls}{}
 \begin{equation}
 	<\lambda>=\frac{3\pi r^4}{16lP} \frac{\Delta P\tau}{V} \sqrt{\frac{\pi RT}{2\mu}}
 \end{equation}

\begin{equation}
	d=\sqrt{\frac{kT}{\sqrt{2}\pi <\lambda>P}}
\end{equation}

\begin{equation}
	\Delta P=\rho g\, \frac{h_1+h_2}{2}
\end{equation}

\begin{equation}
	\varepsilon=\frac{\Delta \rho}{\rho}+\frac{\Delta g}{g}+\frac{\Delta h_1+\Delta h_2}{h_1+h_2}+\frac{\Delta \tau}{\tau}+\frac{\Delta V}{V}
\end{equation}

\paragraph{Результаты прямых измерений и их обработки.}
\begin{itemize}
	\item $T=288\;\mbox{K}$
	\item$P=766\;\mbox{мм рт. ст.}$
	\begin{table}[htb]
	\begin{tabular}{c|c|c|c}
		$h_1$,  см & $h_2$,  см & V,  мл & $\tau$,  с \\
		\hline
		6,7 & 5,4 & 100 & 115 \\
	
		6,7 & 5,4 & 100 & 112 \\
	
		6,7 & 5,4 & 100 & 115 \\
	
	\end{tabular}
\end{table}
\end{itemize}

\paragraph{Расчет результатов косвенных измерений.}
\begin{itemize}
	\item По формуле \hyperlink{formuls}{(3)} находим разность давлений:
	$$ \Delta P=1000\cdot9,8\cdot\frac{6,7\cdot10^{-2}-5,4\cdot10^{-2}}{2}=592,9\;\mbox{(Па)}$$
	\item Среднее время $<\tau>$:
	$$<\tau>=\frac{115+112+115}{3}=114\;\mbox{(c)}$$
	\item Средняя длина $<\lambda>$ свободного пробега молекул воздуха \hyperlink{formuls}{(1)}:
	$$<\lambda>=\frac{3\cdot3,14\cdot(0,5\cdot10^{-3})^4}{16\cdot0,4\cdot766\cdot133,3}\cdot\sqrt{\frac{3,14\cdot8,31\cdot301}{2\cdot29\cdot10^{-3}}}\cdot\frac{592,9\cdot114}{100\cdot10^{-6}}\approx2,24\cdot10^{-7}\;\mbox{(м)}$$
	\item Эффективный диаметр молекул \hyperlink{formuls}{(2)}:
	$$d=\sqrt{\frac{1,38\cdot10^{-23}\cdot301}{\sqrt{2}\cdot3,14\cdot2,24\cdot10^{-7}\cdot766\cdot133,3}}\approx2,02\cdot10^{-10}\;\mbox{(м)}$$
\end{itemize}

\paragraph{Погрешность измерений}
\begin{itemize}
\item
$\Delta\rho=5\;\frac{\mbox{кг}}{\mbox{м}^3}$
%$\Delta g=0,1\;\frac{\mbox{м}}{\mbox{с}^2}$

\end{itemize}
\end{document}
