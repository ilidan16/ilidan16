\documentclass{article}

\usepackage[T2A]{fontenc}
\usepackage[utf8]{inputenc}
\usepackage[russian]{babel}

\usepackage[unicode, colorlinks, linkcolor=blue, citecolor=blue]{hyperref}
\usepackage{amsmath}
\usepackage{graphicx}
\graphicspath{{pictures/}}
\DeclareGraphicsExtensions{.pdf,.png,.jpg}

\usepackage{pgfplots}
\usepackage{tikz}


\begin{document}
\selectlanguage{russian}
	%----------------------------------Шапка-------------------------------------------
\begin{center}
	\includegraphics[scale=0.25]{AU}\\
	{\Large\bfseries Санкт-Петербургский национальный исследовательский Академический университет имени Ж.И.~Алфёрова Российской академии наук}
\end{center}

\begin{center}
	{\large\textbf{Рабочий протокол и отчёт по лабораторной работе № 5}}\\
	Свиридов Фёдор, Александр Слободнюк, Владимир Попов
\end{center}

\begin{center}
	\rule{12cm}{0.4mm}\\
	\large\bfseries{<<Проверка закона Шарля>>}\\
	\rule{12cm}{0.4mm}
\end{center}
%---------------------------------------------------------------------------------------
\paragraph{Цель работы.} Исследование изохорного процесса
\paragraph{Задачи, решаемые при выполении работы.}
\begin{enumerate}
	\item Нагреть газ до 50 градусов цельсия
	\item Начать охлаждение газа
	\item Уменьшать объем, занимаемый газом, после понижения давления на определенную величину
	\item Измерять температуру и давление при этой температуре
	\item Сделать выводы
\end{enumerate}
\paragraph{Объект исследования.} Модель идеального газа
\paragraph{Метод экспериментального исследования.} Поэтапное измерение температуры
\paragraph{Исходные данные.}
Будем считать воздух идеальным газом, тогда:
$$PV=\nu RT $$
$$ P=\frac{\nu R}{V}\,T$$
В нашем опыте количество вещества $\nu$ оставалось примерно постоянным, а вот объём $V$ нам приходилось немного изменять. Опишем несколько последовательных состояний нашей системы.
Пусть в самом начале опыта система находилась в состоянии $F(P_0,V_0,T_0)$
$$ P_0=\frac{\nu R}{V_0}\,T_0\; \xrightarrow{(1)} \; P_1=\frac{\nu R}{V_0}\,T_1 \;\xrightarrow{(2)}\; P_0=\frac{\nu R}{V_0+dV}\,T_1 \;\xrightarrow{(3)} \; P_1=\frac{\nu R}{V_0+dV}\,T_2$$

(1) - изохорный процесс с коэффициентом $ \frac{\nu R}{V_0} $

(2) - возврат к давлению $P_0$ с помощью изменения объёма

(3) - изохорный процесс с коэффициентом $ \frac{\nu R}{V_0+dV} $\\

Таким образом, $\Delta P=C(V)\Delta T$, где C(V) - некоторый коэффициент пропорциональности, который зависит от объёма. Но если пренебречь величиной $dV$, то можно считать, что $\Delta P \sim \Delta T$

\paragraph{Результаты прямых измерений и их обработки.}

\begin{center}
	\centering{Остывание воздуха}\\
	\begin{tabular}{c|c}
	$\Delta P$, кПа&$\Delta T$, $^\circ\mbox{C}$\\
	\hline
	0,2&2,0 \\
	0,2&1,7  \\
	0,2&1,5  \\
	0,2&1,0  \\
	0,2&1,0 \\
	0,2&1,1 \\
	0,2& 1,2 \\
	0,2& 1,0 \\
\end{tabular}
\end{center}
\paragraph{Выводы и анализ результатов.}
Мы провели измерения изменения давления $\Delta P$ и температуры $\Delta T$, чтобы проверить закон Шарля. На основе полученных данных можно не строго говорить о линейной зависимости~$\Delta P \sim \Delta T$. Полученные результаты не очень убедительны, потому что опыт обладает рядом недостатков. Существенной проблемой является то, что установка при повышенном давление пропускает воздух, таким образом, невозможно качественно провести проверку закона Шарля: при постоянном объёме и количестве вещества нагревать (охлаждать) газ и следить за повышением (понижением) давления. Следующим недостатком опыта является малая точность манометра, из-за чего возникают трудности при выравнивание давления до начального состояния $P_0$. Из несущественных недостатков можно отметить, что воздух является не идеальным газом и то, что нам приходилось менять объём на небольшую величину, которой мы пренебрегли.

\end{document}
