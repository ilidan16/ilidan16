\documentclass{article}

\usepackage[T2A]{fontenc}
\usepackage[utf8]{inputenc}
\usepackage[russian,english]{babel}

\usepackage{amsmath}

\usepackage[unicode, colorlinks, linkcolor=blue, citecolor=blue]{hyperref}

\usepackage{graphicx}
\graphicspath{{pictures/}}
\DeclareGraphicsExtensions{.pdf,.png,.jpg}

\usepackage{pgfplots}
\usepackage{tikz}

\begin{document}
	\selectlanguage{russian}
	%----------------------------------Шапка------------------------------------
	\begin{center}
		\includegraphics[scale=0.25]{AU}\\
		{\Large\bfseries Санкт-Петербургский национальный исследовательский Академический университет имени Ж.И.~Алфёрова Российской академии наук}
	\end{center}
	
	\begin{center}
		{\large\textbf{Рабочий протокол и отчёт по лабораторной работе № 3}}\\
		Свиридов Фёдор, Александр Слободнюк, Владимир Попов
	\end{center}
	
	\begin{center}
		\rule{12cm}{0.4mm}\\
		\large\bfseries{<<Определение  средней длины свободного пробега и эффективного диаметра молекул воздуха>>}\\
		\rule{12cm}{0.4mm}
	\end{center}
	%-----------------------------------------------------------------------------
	
	\paragraph{Цель работы.}
	Определить среднюю длину свободного пробега молекул воздуха; определить эффективный диаметр молекул воздуха.
	
	\paragraph{Задачи, решаемые при выполнении работы.}
	\begin{enumerate}
		\item Узнать температуру и атмосферное давление в помещении 
		\item Измерить нужное для расчётов время, за которое в мензурку поступает 100мл воды
		\item Измерить высоту жидкости в колбе до и после вытеснение воды воздухом
		\item  По рабочей формуле рассчитать среднюю длину пробега молекул воздуха
		\item  Рассчитать эффективный диаметр молекул
		\item Сделать выводы
	\end{enumerate}
	
	\paragraph{Объект исследования.}
	Связь макроскопических и микроскопических параметров в рамках данной модели
	
	\paragraph{Метод экспериментального исследования.}
	Измерение скорости вытеснения воды из колбы воздухом
	
	\paragraph{Рабочие формулы и исходные данные.}\hypertarget{formuls}{}
	\begin{equation}
		<\lambda>=\frac{3\pi r^4}{16lP} \frac{\Delta P\tau}{V} \sqrt{\frac{\pi RT}{2\mu}}
	\end{equation}
	
	\begin{equation}
		d=\sqrt{\frac{kT}{\sqrt{2}\pi <\lambda>P}}
	\end{equation}
	
	\begin{equation}
		\Delta P=\rho g\, \frac{h_1+h_2}{2}
	\end{equation}
	
	\begin{equation}
		\varepsilon=\frac{\Delta \rho}{\rho}+\frac{\Delta g}{g}+\frac{\Delta h_1+\Delta h_2}{h_1+h_2}+\frac{\Delta \tau}{\tau}+\frac{\Delta V}{V}
	\end{equation}
	
	\paragraph{Результаты прямых измерений и их обработки.}
	\begin{itemize}
		\item $T=288\;\mbox{K}$
		\item$P=766\;\mbox{мм рт. ст.}$
		\begin{table}[htb]
			\begin{tabular}{c|c|c|c}
				$h_1$,  см & $h_2$,  см & V,  мл & $\tau$,  с \\
				\hline
				6,7 & 5,4 & 100 & 115 \\
				
				6,7 & 5,4 & 100 & 112 \\
				
				6,7 & 5,4 & 100 & 115 \\
			\end{tabular}
		\end{table}
	\end{itemize}
	
	\paragraph{Расчет результатов косвенных измерений.}
	\begin{itemize}
		\item По формуле \hyperlink{formuls}{(3)} находим разность давлений:
		$$ \Delta P=1000\cdot9,8\cdot\frac{6,7\cdot10^{-2}-5,4\cdot10^{-2}}{2}=592,9\;\mbox{(Па)}$$
		\item Среднее время $<\tau>$:
		$$<\tau>=\frac{115+112+115}{3}=114\;\mbox{(c)}$$
		\item Средняя длина $<\lambda>$ свободного пробега молекул воздуха \hyperlink{formuls}{(1)}:
		$$<\lambda>=\frac{3\cdot3,14\cdot(0,5\cdot10^{-3})^4}{16\cdot0,4\cdot766\cdot133,3}\cdot\sqrt{\frac{3,14\cdot8,31\cdot301}{2\cdot29\cdot10^{-3}}}\cdot\frac{592,9\cdot114}{100\cdot10^{-6}}\approx2,24\cdot10^{-7}\;\mbox{(м)}$$
		\item Эффективный диаметр молекул \hyperlink{formuls}{(2)}:
		$$d=\sqrt{\frac{1,38\cdot10^{-23}\cdot301}{\sqrt{2}\cdot3,14\cdot2,24\cdot10^{-7}\cdot766\cdot133,3}}\approx2,02\cdot10^{-10}\;\mbox{(м)}$$
	\end{itemize}
	
	\paragraph{Погрешность измерений.}
	\begin{itemize}
		\item $\Delta\rho=5\;\frac{\mbox{кг}}{\mbox{м}^3}$
		\item$\Delta g=0,1\;\frac{\mbox{м}}{\mbox{с}^2}$
		\item$\Delta h_1=\Delta h_2=0,1\;\mbox{см}$
		\item$\Delta \tau=1\;\mbox{c}$
		\item$\Delta V=5\;\mbox{мл}$
	\end{itemize}
	Пользуясь формулой \hyperlink{formuls}{(4)}, находим относительную погрешность $\varepsilon$:
	$$\varepsilon=\frac{5}{1000}+\frac{0,1}{9,8}+\frac{0,1\cdot10^{-2}+0,1\cdot10^{-2}}{6,7\cdot10^{-2}+5,4\cdot10^{-2}}+\frac{1}{114}+\frac{5\cdot10^{-6}}{100\cdot10^{-6}}\approx0,0905$$
	Абсолютная погрешность средней длины $<\lambda>$  свободного пробега:
	$$\Delta \lambda=\varepsilon\lambda=0,0905\cdot2,24\cdot10^{-7}=2,0\cdot10^{-8}\;\mbox{(м)}$$
	Погрешность диаметра оценим по формуле:
	$$ \Delta d=\frac{1}{2}\sqrt{\frac{kT}{\sqrt{2}\pi P}}\;\lambda^{-\frac{3}{2}}\;\Delta\lambda$$
	$$ \Delta d=9\cdot10^{-12}\;\mbox{м}$$
	\paragraph{Окончательные результаты.}
	\begin{itemize}
		\item Средняя длина $<\lambda>$ свободного пробега молекул воздуха:
		\begin{center}
			\fbox{$<\lambda>\;=(2,24\pm0,20)\cdot10^{-7}\;\mbox{м}$}
		\end{center}
		
		\item Эффективный диаметр $d$ молекул:
		\begin{center}
			\fbox{$d\;=(2,02\pm0,09)\cdot10^{-10}\;\mbox{м}$}
		\end{center}
		
	\end{itemize}
	
	\paragraph{Выводы и анализ результатов.}
	Мы косвенно измерили среднюю длину свободного пробега молекул воздуха и их эффективный диаметр. Полученные результаты по порядку величин хорошо согласуются с ожидаемыми. Например, в приложении учебника Иродова И. Е.~\cite{1} указано, что диаметр молекул воздуха  равен $3,5\cdot10^{-10}\; \mbox{(м)}$. Стоит отметить, что данный способ определения диаметра и свободного пробега служит только для оценки, так как он основывается на модели идеального газа и на сферической симметрии молекул.
	
	\begin{thebibliography}{3}
		\bibitem{1}
		Иродов И. Е. Физика макросистем. Основные законы. 2015. 210 с.
	\end{thebibliography}
\end{document}